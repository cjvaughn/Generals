% Copyright 2004 by Till Tantau <tantau@users.sourceforge.net>.
%
% In principle, this file can be redistributed and/or modified under
% the terms of the GNU Public License, version 2.
%
% However, this file is supposed to be a template to be modified
% for your own needs. For this reason, if you use this file as a
% template and not specifically distribute it as part of a another
% package/program, I grant the extra permission to freely copy and
% modify this file as you see fit and even to delete this copyright
% notice. 

\documentclass{beamer}
\usepackage{graphicx}

% There are many different themes available for Beamer. A comprehensive
% list with examples is given here:
% http://deic.uab.es/~iblanes/beamer_gallery/index_by_theme.html
% You can uncomment the themes below if you would like to use a different
% one:
%\usetheme{AnnArbor}
%\usetheme{Antibes}
%\usetheme{Bergen}
%\usetheme{Berkeley}
%\usetheme{Berlin}
%\usetheme{Boadilla} %Not bad
%\usetheme{boxes}
%\usetheme{CambridgeUS}
%\usetheme{Copenhagen}
%\usetheme{Darmstadt}
%\usetheme{default}
%\usetheme{Frankfurt}
%\usetheme{Goettingen}
%\usetheme{Hannover}
%\usetheme{Ilmenau}
%\usetheme{JuanLesPins}
%\usetheme{Luebeck}
\usetheme{Madrid}
%\usetheme{Malmoe}
%\usetheme{Marburg}
%\usetheme{Montpellier}
%\usetheme{PaloAlto}
%\usetheme{Pittsburgh}
%\usetheme{Rochester}
%\usetheme{Singapore}
%\usetheme{Szeged}
%\usetheme{Warsaw}
\DeclareMathOperator*{\argmin}{arg\,min}

\title{Synchronization of Circadian Oscillators}

% A subtitle is optional and this may be deleted
\subtitle{Mean Field Game Formulation}

%\author{Christy Graves}
\author[Christy Graves]{Christy Graves\\{\small Advisor: Ren\'{e} Carmona}}
% - Give the names in the same order as the appear in the paper.
% - Use the \inst{?} command only if the authors have different
%   affiliation.

\institute[Princeton University] % (optional, but mostly needed)
{
	Program in Applied and Computational Mathematics\\
	Princeton University
}
% - Use the \inst command only if there are several affiliations.
% - Keep it simple, no one is interested in your street address.

\date{Generals, May 4 2017}
% - Either use conference name or its abbreviation.
% - Not really informative to the audience, more for people (including
%   yourself) who are reading the slides online

%\subject{Theoretical Computer Science}
% This is only inserted into the PDF information catalog. Can be left
% out. 

% If you have a file called "university-logo-filename.xxx", where xxx
% is a graphic format that can be processed by latex or pdflatex,
% resp., then you can add a logo as follows:

% \pgfdeclareimage[height=0.5cm]{university-logo}{university-logo-filename}
% \logo{\pgfuseimage{university-logo}}

% Delete this, if you do not want the table of contents to pop up at
% the beginning of each subsection:
\AtBeginSection[]
{
  \begin{frame}<beamer>{Outline}
    \tableofcontents[currentsection,currentsubsection]
  \end{frame}
}

% Let's get started
\begin{document}

\begin{frame}
  \titlepage
\end{frame}

\begin{frame}{Outline}
  \tableofcontents
  % You might wish to add the option [pausesections]
\end{frame}

% Section and subsections will appear in the presentation overview
% and table of contents.
\section{Introduction}
%\frame{\tableofcontents[currentsection]}

\begin{frame}{Circadian Oscillators}
	\begin{itemize}
		\item {
			Cells in Suprachiasmatic Nucleus (SCN) responsible for circadian rhythm.
		}
		\item Order of $10^4$ neuronal cells.
		\item Each cell exhibits oscillatory dynamics.
		\item Each cell has a preferred frequency $\omega$ that varies from cell to cell.
		\item $\mathbb{E}(\omega)=\frac{2\pi}{24.5}$
		\item Cells want to synchronize with each other, as well any external sources (e.g. the sun).
	\end{itemize}
\end{frame}

\section{State of the Art}

\subsection{Mean Field Games}

\begin{frame}{State of the Art: Mean Field Games}
	Strategy:
	\begin{itemize}
				\item {
					Fix flow of measures: $\mu=(\mu_t)_{0 \leq t \leq T}$
				}
				\item {
					Solve stochastic control problem: find $\alpha^{\mu} \in \argmin J(\alpha,X^{\alpha},\mu)$
				}
				\item {
					Find fixed point: $\mu$ s.t. $\mathcal{L}(X_t^{\alpha^{\mu}})=\mu_t$
				}
	\end{itemize}
\end{frame}

\begin{frame}{State of the Art: Mean Field Games}{Existence \& Uniqueness}
	\begin{itemize}
		\item {
			Cauchy-Lipschitz theory only holds for small T.
		}
		\item {
			Existence
			\begin{itemize}
				\item In general, need Lipschitz coefficients and cost functions, non-degenerate diffusion coefficient.
				\item Relies on Schauder's Fixed Point Theorem.
			\end{itemize}
		}
		\item {
			Uniqueness
			\begin{itemize}
				\item Assumming no common noise, in general, need monotone cost functions.
				\begin{itemize}
				\item Lasry Lions Monotonicity
				\item $L$-Monotonicity
				\end{itemize}
			\end{itemize}
		}
	\end{itemize}
\end{frame}

\begin{frame}{State of the Art: Mean Field Games}{Some Results}
	\begin{itemize}
		\item {
			Linear-Quadratic Games.
		}
		\begin{itemize}
			\item Stochastic maximum approach leads to matrix Ricatti equation.
		\end{itemize}
		\item Some results when cost is local.
		\begin{itemize}
			\item Cost is local and linear: Swiecicki et. al. \textit{An [imaginary time] Schr{\"o}dinger approach to mean field games.}
		\end{itemize}
		\item With common noise: Cardaliaguet et. al. \textit{The master equation and the convergence problem in mean field games.}
		\item Nonlocal cost of particular form: Graber \& Bensoussan \textit{Existence and Uniqueness of Solutions for Betrand and Cournot mean field games.}
	\end{itemize}
\end{frame}

\begin{frame}{State of the Art: Mean Field Games}{Open Problems}
	\begin{itemize}
		\item {
			More general existence and uniqueness results.
		}
		\item {
			Problems with non-convex cost functions.
		}
		\item {
			Problems with non-local cost functions.
		}
	\end{itemize}
\end{frame}

\subsection{Circadian Oscillators: Previous Work}

\begin{frame}{State of the Art: Circadian Oscillators}{Previous Work}
	Two groups:
	\begin{itemize}
		\item Lu, Carde\~na, Lee, Antonsen, Girvan, \& Ott at University of Maryland College Park, \textit{Resynchronization of circadian oscillators and the east-west asymmetry of jet-lag}, 2016
		\item Yin, Mehta, Meyn, \& Shanbhag at University of Illinois at Urbana-Champaign, \textit{Synchronization of coupled oscillators is a game}, 2012
	\end{itemize}
\end{frame}

\begin{frame}{State of the Art: Circadian Oscillators}{Previous Work: Lu, et. al.}
	\begin{itemize}
		\item Large number of oscillators following Kuramoto model:
		\begin{equation}
		\frac{d\theta_i}{dt}=\omega_i+\frac{K}{N} \sum_{i=1}^N \sin(\theta_j-\theta_i)+F \sin (\omega_S t+p(t)-\theta_i)
		\end{equation}
		\item Define order parameter, $z(t)$:
		\begin{equation}
		z(t)=\frac{1}{N}\sum_{i=1}^N e^{i[\theta_i(t)-\omega_St-p(t)]}
		\end{equation}
		\item Then assuming $\omega_i \sim$ Cauchy$(\omega_0,\Delta)$, then $z$ satisfies:
		\begin{equation}
		\dot{z}=\frac{1}{2} \left[(Kz+F)-z^2 (Kz+F)^* \right]- (\Delta+i(\omega_S-\omega_0))z
		\end{equation}
	\end{itemize}
\end{frame}

\begin{frame}{State of the Art: Circadian Oscillators}{Previous Work: Lu, et. al., continued}
	\begin{itemize}
		\item An individual is entrained to their local time zone if $z$ is at a stable fixed point.
		\item Changing your time zone corresponds to a rotation of z by $\Delta p=p_1-p_2$.
		\item The time to return to the stable fixed point can be identified as the time to recover from jet-lag.
		\item The authors were interested in understanding the East-West asymmetry of jet-lag.
	\end{itemize}
\end{frame}

\begin{frame}{State of the Art: Circadian Oscillators}{Previous Work: Lu, et. al., Main Results}
	\begin{center}
	\includegraphics[scale=0.2]{Lu_1}
	\end{center}
	Taken from Lu, et. al.
\end{frame}

\begin{frame}{State of the Art: Circadian Oscillators}{Previous Work: Lu, et. al., Main Results}
	\begin{center}
		\includegraphics[scale=0.25]{Lu_2}
	\end{center}
	Taken from Lu, et. al.
\end{frame}

\begin{frame}{State of the Art: Circadian Oscillators}{Previous Work: Lu, et. al., Summary}
	\begin{itemize}
		\item East-West asymmetry of jetlag explained by $\omega_0<\frac{2 \pi}{24}=\omega_S$
		\item Oscillators are not solving an optimization problem.
	\end{itemize}
\end{frame}

\begin{frame}{State of the Art: Circadian Oscillators}{Previous Work: Yin, et. al.}
		Each oscillator evolves it's phase according to:
		\begin{equation}
			d\theta_t=(\omega_0+\alpha)dt+\sigma dW_t
		\end{equation}
		where $\alpha(t)$ is chosen to minimize the long run average (LRA) cost:
		\begin{equation}
		\eta=\limsup_{T \rightarrow \infty} \frac{1}{T}\int_0^T \left(\frac{R}{2} \alpha^2+\bar{c}(t,\theta,\mu_t) \right) dt
		\end{equation}
		where
		\begin{equation}
		\bar{c}(t,\theta,\mu_t)=\int_0^{2\pi}\frac{1}{2}\sin^2\left(\frac{\theta-\theta'}{2}\right) d\mu_t(\theta')
		\end{equation}
\end{frame}

\begin{frame}{State of the Art: Circadian Oscillators}{Previous Work: Yin, et. al., continued}
	Let $\eta^*$ be the ergodic average cost:
	\begin{equation}
	\eta^*=\lim_{T \rightarrow \infty} \frac{1}{T} \int_0^T \int_0^{2\pi} \left[\frac{1}{2R}(\partial_\theta V)^2+\bar{c}(t,\theta,\mu(t,\cdot))+c_{sun}(t,\theta) \right] d\mu(t,\theta)
	\end{equation}
	
	The solution is given by the HJB and Kolmogorov equations:
	
	\begin{equation}
	\partial_t V+\omega \partial_\theta V=-\frac{\sigma^2}{2}\partial_{\theta \theta}^2 V+\eta^*+\frac{1}{2R}(\partial_\theta V)^2-\bar{c}(t,\theta,\mu(t,\cdot))-c_{sun}(t,\theta)
	\label{HJB}
	\end{equation}
	
	\begin{equation}
	\partial_t \mu+\omega \partial_{\theta} \mu= \frac{1}{R} \partial_{\theta}\left[\mu(\partial_\theta V) \right]+ \frac{\sigma^2}{2} \partial_{\theta \theta}^2 \mu
	\label{Kolmogorov}
	\end{equation}
\end{frame}

\begin{frame}{State of the Art: Circadian Oscillators}{Previous Work: Yin, et. al., continued}
Look for two types of solutions:
\begin{itemize}
	\item Time independent: incoherence solution
	\begin{equation}
	\begin{split}
		V(t,\theta)&=0 \\
		\mu(t,\theta)&=\frac{1}{2\pi}
	\end{split}
	\end{equation}
	\item Time-periodic solutions: traveling waves
\end{itemize}
\end{frame}

\begin{frame}{State of the Art: Circadian Oscillators}{Previous Work: Yin, et. al., Main Results}
	\begin{itemize}
		\item Analysis of linearization of perturbation about the incoherence solution.
		\item For $R>R_c$, the incoherence solution is linearly asymptotically stable.
		\item From $R=R_c$ bifurcates a family of (non-constant) traveling wave solutions.
	\end{itemize}
\end{frame}

\begin{frame}{State of the Art: Circadian Oscillators}{Open Problems}
	\begin{itemize}
		\item { Mean field game formulation for the synchronization of circadian oscillators in the presence of an external source.
		}
	\end{itemize}
\end{frame}

\section{My Progress}

\subsection{Problem Formulation}

\begin{frame}{Problem Formulation}
		Each oscillator evolves it's phase according to:
		\begin{equation}
		d\theta_t=(\omega_0+\alpha_t)dt+\sigma dW_t
		\end{equation}
		where $\alpha_t$ is chosen to minimize the long run average (LRA) cost:
		\begin{equation}
		\eta=\limsup_{T \rightarrow \infty} \frac{1}{T}\int_0^T \left(\frac{R}{2} \alpha_t^2+\bar{c}(t,\theta,\mu_t)+c_{sun}(t,\theta_t) \right) dt
		\end{equation}
		where
		\begin{equation}
		\bar{c}(t,\theta,\mu_t)=\int_0^{2\pi}\frac{1}{2}\sin^2\left(\frac{\theta-\theta'}{2}\right) d\mu_t(\theta')
		\end{equation}
		and
		\begin{equation}
		c_{sun}(t,\theta)=\frac{F}{2}\sin^2\left(\frac{\omega_St-\theta}{2}\right)
		\end{equation}
\end{frame}

\begin{frame}{PDEs}
	Let $\eta^*$ be the ergodic average cost:
	\begin{equation}
	\eta^*=\lim_{T \rightarrow \infty} \frac{1}{T} \int_0^T \int_0^{2\pi} \left[\frac{1}{2R}(\partial_\theta V)^2+\bar{c}(t,\theta,\mu(t,\cdot))+c_{sun}(t,\theta) \right] d\mu(t,\theta)
	\end{equation}
	
	The solution is given by the HJB and Kolmogorov equations:
	
	\begin{equation}
	\partial_t V+\omega \partial_\theta V=-\frac{\sigma^2}{2}\partial_{\theta \theta}^2 V+\eta^*+\frac{1}{2R}(\partial_\theta V)^2-\bar{c}(t,\theta,\mu(t,\cdot))-c_{sun}(t,\theta)
	\label{HJB}
	\end{equation}
	
	\begin{equation}
	\partial_t \mu+\omega \partial_{\theta} \mu= \frac{1}{R} \partial_{\theta}\left[\mu(\partial_\theta V) \right]+ \frac{\sigma^2}{2} \partial_{\theta \theta}^2 \mu
	\label{Kolmogorov}
	\end{equation}
\end{frame}

\subsection{Numerical Approach}

\begin{frame}{Numerical Approach}
	Discretize the problem using finite differences.
	\begin{itemize}
			\item {
				Goal: find grid functions $V(t_k,x_i,v_j)$ and $\mu_{t_k}(x_i,v_j)$ solving finite difference equations for HJB and Kolmogorov.
			}
			\item {
				Step 0: start with some guess for $\mu=(\mu_{t_k}(x_i,v_j))_{0 \leq t_k \leq T}$
			}
			\item {
				Step 1: Given $\mu$ and $V(T,x)$, solve finite difference HJB explicitly backwards in time for $V$, which gives us optimal $\alpha$.
			}
			\item {
				Step 2: Given $\alpha$ and $\mu^0$, solve finite difference Kolmogorov explicitly forwards in time for $\mu '$.
			}
			\item {
				Repeat Steps 1 and 2 until $\mu \approx \mu '$.
			}
	\end{itemize}
\end{frame}

\begin{frame}{Numerical Approach: Stability and Accuracy}
		\begin{itemize}
			\item CFL condition:
			\begin{equation} 
			\Delta t \leq \frac{1}{2(\frac{\sigma^2}{\Delta x^2}+\frac{|b|}{\Delta x})}
			\end{equation}
			\item Upwind scheme:
			\begin{equation}
			[Lf]=b \cdot \frac{\partial f}{\partial x} + \frac{\sigma^2}{2}\frac{\partial^2 f}{\partial x^2}
			\end{equation}
			\begin{equation}
			[Lf](i)=\sum_{j} L(i,j) f(x_j)
			\end{equation}
			Choose forward or backward differences such that:
			\begin{enumerate}
				\item $L(i,j)\geq 0$ when $i \neq j$.
				\item $L(i,i)=-\sum_{j\neq i} L(i,j)$
			\end{enumerate}
			
		\end{itemize}
\end{frame}
	
\begin{frame}{Numerical Approach: Stability of $\mu$}
		\begin{itemize}
			\item CFL + Upwind $\rightarrow$ Solution to Kolmogorov is a prob. measure, i.e.
			\begin{equation}
			\begin{split}
			\mu(t_n,x_i) & \geq 0 \\
			\sum_i \mu(t_n,x_i) &= 1
			\end{split}
			\end{equation}
			\item Proof:
			\begin{equation}
			\frac{\mu(t_{n+1},x_i)-\mu(t_{n},x_i)}{\Delta t}=\sum_{j} L(j,i) \mu(t_n,x_j)
			\end{equation}
			\begin{equation}
			\begin{split}
			\mu(t_{n+1},x_i)&=\mu(t_{n},x_i)+ \Delta t \cdot L(i,i)\mu(t_n,x_i) + \Delta t \cdot \sum_{j\neq i} L(j,i) \mu(t_n,x_j)  \\
			& \geq \mu(t_{n},x_i) (1+ \Delta t \cdot L(i,i))
			\end{split}
			\end{equation}
			$\mu(t_{n+1},x_i) \geq 0$ if $\Delta t \cdot| L(i,i)| \leq 1$ (CFL Condition)
		\end{itemize}
\end{frame}
	
\begin{frame}{Numerical Approach: Stability of $\mu$ (continued)}
		\begin{equation}
		\begin{split}
		\sum_i \mu(t_{n+1},x_i)=& \sum_i \mu(t_{n},x_i)+\Delta t \cdot \sum_i \sum_{j} L(j,i) \mu(t_n,x_j) \\
		& = 1 + \Delta t \cdot \sum_j \Big(\sum_i L(j,i)\Big) \mu(t_n,x_j) \\
		& = 1
		\end{split}
		\end{equation}
\end{frame}
	
\begin{frame}{Numerical Approach}
ToDo:

describe the 3 step process:
1) Start from $\mu$ uniform, $V(T)=0$
2) Start from $\mu$ traveling wave from 1)
3) Start from $\mu$ traveling wave from 1) and also with V(T) from 2)


ToDo:
Describe how to deal with eta and ergodic form of cost
\end{frame}

\subsection{Numerical Results}

\begin{frame}{Numerical Results}{Traveling Waves}
	\begin{itemize}
		\item {
			My first point.
		}
		\item {
			My second point.
		}
	\end{itemize}
\end{frame}

\begin{frame}{Numerical Results}{Re-synchronization After Travel}
	\begin{itemize}
		\item {
			My first point.
		}
		\item {
			My second point.
		}
	\end{itemize}
\end{frame}

\begin{frame}{Numerical Results}{East versus West}
	\begin{itemize}
		\item {
			My first point.
		}
		\item {
			My second point.
		}
	\end{itemize}
\end{frame}

\subsection{Comparison with Previous Work}

\begin{frame}{Comparison with Previous Work}
	\begin{itemize}
		\item {
			My first point.
		}
		\item {
			My second point.
		}
	\end{itemize}
\end{frame}

\subsection{Analytical Progress}

\begin{frame}{Analytical Progress}
	\begin{itemize}
		\item {
			My first point.
		}
		\item {
			My second point.
		}
	\end{itemize}
\end{frame}


\section{Next Steps}
\begin{frame}{Next Steps: Numerics}
	\begin{itemize}
		\item {
			My first point.
		}
		\item {
			My second point.
		}
	\end{itemize}
\end{frame}

\begin{frame}{Next Steps: Analytically}
	\begin{itemize}
		\item {
			My first point.
		}
		\item {
			My second point.
		}
	\end{itemize}
\end{frame}

% Placing a * after \section means it will not show in the
% outline or table of contents.
\section*{Summary}
\begin{frame}{Summary}
	\begin{itemize}
		\item {
			My first point.
		}
		\item {
			My second point.
		}
	\end{itemize}
\end{frame}

\section*{Questions}
\begin{frame}{Questions?}
	\begin{center}
		Thanks for listening! Any questions?
	\end{center}
\end{frame}

\end{document}


