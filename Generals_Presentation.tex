% Copyright 2004 by Till Tantau <tantau@users.sourceforge.net>.
%
% In principle, this file can be redistributed and/or modified under
% the terms of the GNU Public License, version 2.
%
% However, this file is supposed to be a template to be modified
% for your own needs. For this reason, if you use this file as a
% template and not specifically distribute it as part of a another
% package/program, I grant the extra permission to freely copy and
% modify this file as you see fit and even to delete this copyright
% notice. 

\documentclass{beamer}

% There are many different themes available for Beamer. A comprehensive
% list with examples is given here:
% http://deic.uab.es/~iblanes/beamer_gallery/index_by_theme.html
% You can uncomment the themes below if you would like to use a different
% one:
%\usetheme{AnnArbor}
%\usetheme{Antibes}
%\usetheme{Bergen}
%\usetheme{Berkeley}
%\usetheme{Berlin}
%\usetheme{Boadilla}
%\usetheme{boxes}
%\usetheme{CambridgeUS}
%\usetheme{Copenhagen}
%\usetheme{Darmstadt}
%\usetheme{default}
%\usetheme{Frankfurt}
%\usetheme{Goettingen}
%\usetheme{Hannover}
%\usetheme{Ilmenau}
%\usetheme{JuanLesPins}
%\usetheme{Luebeck}
\usetheme{Madrid}
%\usetheme{Malmoe}
%\usetheme{Marburg}
%\usetheme{Montpellier}
%\usetheme{PaloAlto}
%\usetheme{Pittsburgh}
%\usetheme{Rochester}
%\usetheme{Singapore}
%\usetheme{Szeged}
%\usetheme{Warsaw}
\DeclareMathOperator*{\argmin}{arg\,min}

\title{Numerical Solutions to Mean Field Games}

% A subtitle is optional and this may be deleted
\subtitle{Models of Consensus}

%\author{Christy Graves}
\author[Christy Graves]{Christy Graves\\{\small Advisor: Ren\'{e} Carmona}}
% - Give the names in the same order as the appear in the paper.
% - Use the \inst{?} command only if the authors have different
%   affiliation.

\institute[Princeton University] % (optional, but mostly needed)
{
	Program in Applied and Computational Mathematics\\
	Princeton University
}
% - Use the \inst command only if there are several affiliations.
% - Keep it simple, no one is interested in your street address.

\date{Generals Exam, May 4 2017}
% - Either use conference name or its abbreviation.
% - Not really informative to the audience, more for people (including
%   yourself) who are reading the slides online

\subject{Theoretical Computer Science}
% This is only inserted into the PDF information catalog. Can be left
% out. 

% If you have a file called "university-logo-filename.xxx", where xxx
% is a graphic format that can be processed by latex or pdflatex,
% resp., then you can add a logo as follows:

% \pgfdeclareimage[height=0.5cm]{university-logo}{university-logo-filename}
% \logo{\pgfuseimage{university-logo}}

% Delete this, if you do not want the table of contents to pop up at
% the beginning of each subsection:
\AtBeginSection[]
{
  \begin{frame}<beamer>{Outline}
    \tableofcontents[currentsection,currentsubsection]
  \end{frame}
}

% Let's get started
\begin{document}

\begin{frame}
  \titlepage
\end{frame}

\begin{frame}{Outline}
  \tableofcontents
  % You might wish to add the option [pausesections]
\end{frame}

% Section and subsections will appear in the presentation overview
% and table of contents.
\section{Introduction}
%\frame{\tableofcontents[currentsection]}
\begin{frame}{Brief Intro to MFG}{N player stochastic game}
		$N$ players each make a decision (or control) $\alpha_t^i \in \mathbb{A}$ at time $t$ in an attempt to minimize their cost functions, $J^i$. For example:
		\begin{equation}
		\begin{split}
		J^i(\alpha^i,X^{1,\alpha},...X^{N,\alpha})=&\mathbb{E}\Bigg[\int_{0}^{T}f^i(t,\alpha_t^i,X_t^{1,\alpha},...X_t^{N,\alpha})dt \\
		&+g^i(X_T^{1,\alpha},...X_T^{N,\alpha}) \Bigg]
		\end{split}
		\end{equation}
		where $X_t^{i,\alpha}$ is the state of player $i$ at time $t$, given by:
		\begin{equation}
		dX_t^i=b(t,\alpha_t^i,X_t^{i,\alpha},\bar{\mu}_t)dt+\sigma(t,\alpha_t^i,X_t^{i,\alpha},\bar{\mu}_t)dW_t
		\end{equation}
		and $f^i$ and $g^i$ are the running costs and terminal costs, respectively.
\end{frame}

	\begin{frame}{Brief Intro to MFG}{Moving to Mean Field}
		\begin{itemize}
			\item {
				Assume players are symmetric.
			}
			\item {
				Consider the limit $N \rightarrow \infty$ and consider the mean field game.
		    }
			\item {
				Generic player has control $\alpha_t \in \mathbb{A}$ and state $X_t^{\alpha}$.
			}
			\item {
				Distribution of players' states over time: $\mu=(\mu_t)_{0 \leq t \leq T}$.
			}
			\item {
				Cost for generic player:
				\begin{equation}
				J(\alpha,X^{\alpha},\mu)=\mathbb{E}\Bigg[\int_{0}^{T}f(t,\alpha_t,X_t^{\alpha},\mu_t)dt+g(X_T^{\alpha},\mu_T) \Bigg]
				\end{equation}
			}
			\pause
			\item {
				Goal: find analog to Nash equilibrium
			}
		\end{itemize}
	\end{frame}



\begin{frame}{Two Consensus Problems}
	\begin{itemize}
		\item {
			Flocking: large number of birds reaching consensus on flocking velocity
		}
		\item {
			Circadian Rhythm: large number of neuronal oscillators synchronizing with each other and external stimuli (e.g. the sun)
		}
	\end{itemize}
\end{frame}

\section{State of the Art}

\subsection{Mean Field Games}

\begin{frame}{State of the Art: Mean Field Games}{Analytic Approach}
	Strategy:
	\begin{itemize}
				\item {
					Fix flow of measures: $\mu=(\mu_t)_{0 \leq t \leq T}$
				}
				\item {
					Solve stochastic control problem: find $\alpha^{\mu} \in \argmin J(\alpha,X^{\alpha},\mu)$
				}
				\item {
					Find fixed point: $\mu$ s.t. $\mathcal{L}(X_t^{\alpha^{\mu}})=\mu_t$
				}
	\end{itemize}
\end{frame}
		
\begin{frame}{State of the Art: Mean Field Games}{Analytic Approach: Stochastic Control Problem}
	\begin{itemize}
				\item {
					Define the value function:
					\begin{equation}
					\begin{split}
					V(t,x)=\inf_{\alpha \in \mathbb{A}}\mathbb{E}&\Bigg[\int_{t}^{T}f(s,\alpha_s,X_s^{\alpha},\mu_s)ds \\
					&+g(X_T^{\alpha},\mu_T) \mid X_t = x \Bigg]
					\end{split}
					\end{equation}
				}
			\end{itemize}
\end{frame}
		
\begin{frame}{State of the Art: Mean Field Games}{Analytic Approach: Stochastic Control Problem: HJB Equation}
	\begin{itemize}
				\item {
					The value function satisfies the HJB equation:
					\begin{block}{Hamilton Jacobi Bellman Equation}
						\begin{equation}
						\begin{split}
						\partial_tV(t,x)+ H(t,x, \mu_t,\partial_x V,\hat{\alpha}(t,x,\mu_t, \partial_x V))&=0 \\
						V(T,x)&=g(x,\mu_T)
						\end{split}
						\end{equation}
					\end{block}
					where  $H$ is the Hamiltonian, and $\hat{\alpha}$ is chosen to minimize the Hamiltonian.
				}
				\item {
					Derivation from the dynamic programming principle and It\^{o}'s formula
				}
			\end{itemize}
\end{frame}
		
\begin{frame}{State of the Art: Mean Field Games}{Analytic Approach: Kolmogorov (or Fokker-Planck) Equation}
	\begin{itemize}
				\item {
					Solving HJB gives us an optimal control: $\alpha(t,x)$.
				}
				\item {
					If all players use the optimal control, what is $\mathcal{L}(X_t^{\alpha})=:\mu_t$?
				}
				\item {
					This is given by the Kolmogorov (or Fokker-Planck) equation:
					\begin{block}{Kolmogorov Equation}
						\begin{equation}
						\begin{split}
						\partial_t \mu_t-\frac{1}{2} trace(\sigma \sigma^T \partial_{xx}^2 \mu_t)& \\
						+ div(b(t,x, \mu_t,\hat{\alpha}(t,x,\mu_t, \partial_x V))\cdot \mu_t)&=0 \\
						\mu_0&= \mu^0   
						\end{split}
						\end{equation}
					\end{block}
				}
				\item {
					Derivation from It\^{o}'s formula and integration by parts
				}
			\end{itemize}
\end{frame}
		
\begin{frame}{State of the Art: Mean Field Games}{Analytic Approach: Strategy Summary}
	\begin{itemize}
				\item {
					Goal: find $V, \mu$ solving HJB and Kolmogorov.
				}
				\item {
					Step 0: start with some guess for $\mu=(\mu_t)_{0 \leq t \leq T}$
				}
				\item {
					Step 1: Given $\mu$ and $V(T,x)$, solve HJB for $V$, which gives us optimal $\alpha$.
				}
				\item {
					Step 2: Given $\alpha$ and $\mu^0$, solve Kolmogorov for $\mu '$.
				}
				\item {
					Repeat Steps 1 and 2 until $\mu = \mu '$.
				}
				\pause
				\item {
					Note: existence, uniqueness, or convergence are not guaranteed. In general, need convexity of $\mathbb{A}$, $f$, and $g$.
				}
				\pause
				\item {
					Another note: this system of PDEs is highly coupled, nonlinear, with one PDE in the forward direction and the other in the backward direction.
				}
	\end{itemize}
\end{frame}

\begin{frame}{State of the Art: Mean Field Games}{Existence \& Uniqueness}
	\begin{itemize}
		\item {
			My first point.
		}
		\item {
			My second point.
		}
	\end{itemize}
\end{frame}

\begin{frame}{State of the Art: Mean Field Games}{Open Problems}
	\begin{itemize}
		\item {
			My first point.
		}
		\item {
			My second point.
		}
	\end{itemize}
\end{frame}

\subsection{Flocking: Previous Work}

\begin{frame}{State of the Art: Flocking}{Problem Formulations}
	\begin{itemize}
		\item {
			My first point.
		}
		\item {
			My second point.
		}
	\end{itemize}
\end{frame}

\begin{frame}{State of the Art: Flocking}{Previous Work}
	Cucker-Smale
	\begin{equation}
				\begin{split}
				dx_t&=v_tdt\\
				dv_t&=\alpha_t dt %ToDo
				\end{split}
	\end{equation}
	\begin{itemize}
		\item {
			My first point.
		}
		\item {
			My second point.
		}
	\end{itemize}
\end{frame}

\begin{frame}{State of the Art: Flocking}{Open Problems}
	\begin{itemize}
		\item {
			My first point.
		}
		\item {
			My second point.
		}
	\end{itemize}
\end{frame}

\section{My Progress}

\subsection{Flocking Problem Formulation}

\begin{frame}{Flocking Problem Formulation}
	\begin{itemize}
		\item {
			My first point.
		}
		\item {
			My second point.
		}
	\end{itemize}
\end{frame}

\subsection{Deriving Stable Scheme \& Boundary Conditions}

\begin{frame}{Numerical Approach}
	\begin{itemize}
		\item {
			My first point.
		}
		\item {
			My second point.
		}
	\end{itemize}
\end{frame}

\begin{frame}{Stability \& Boundary Conditions}
	\begin{itemize}
		\item {
			My first point.
		}
		\item {
			My second point.
		}
	\end{itemize}
\end{frame}

\subsection{Numerical Results}

\begin{frame}{Numerical Results}
	\begin{itemize}
		\item {
			My first point.
		}
		\item {
			My second point.
		}
	\end{itemize}
\end{frame}
\subsection{Analytical Progress}

\begin{frame}{Analytical Progress}
	\begin{itemize}
		\item {
			My first point.
		}
		\item {
			My second point.
		}
	\end{itemize}
\end{frame}

\section{Future Directions}

\subsection{Numerics}

\begin{frame}{Future Directions}{Numerics}
	\begin{itemize}
		\item {
			My first point.
		}
		\item {
			My second point.
		}
	\end{itemize}
\end{frame}

\subsection{Ideas for Proving Conjecture}

\begin{frame}{Future Directions}{Ideas for Proving Conjecture}
	\begin{itemize}
		\item {
			My first point.
		}
		\item {
			My second point.
		}
	\end{itemize}
\end{frame}

\subsection{New Problem: Circadian Oscillators}

\begin{frame}{Future Directions}{New Problem: Circadian Oscillators}
	\begin{itemize}
		\item {
			My first point.
		}
		\item {
			My second point.
		}
	\end{itemize}
\end{frame}

% Placing a * after \section means it will not show in the
% outline or table of contents.
\section*{Summary}
\begin{frame}{Summary}
	\begin{itemize}
		\item {
			My first point.
		}
		\item {
			My second point.
		}
	\end{itemize}
\end{frame}

\end{document}


