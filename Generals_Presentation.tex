% Copyright 2004 by Till Tantau <tantau@users.sourceforge.net>.
%
% In principle, this file can be redistributed and/or modified under
% the terms of the GNU Public License, version 2.
%
% However, this file is supposed to be a template to be modified
% for your own needs. For this reason, if you use this file as a
% template and not specifically distribute it as part of a another
% package/program, I grant the extra permission to freely copy and
% modify this file as you see fit and even to delete this copyright
% notice. 

\documentclass{beamer}

% There are many different themes available for Beamer. A comprehensive
% list with examples is given here:
% http://deic.uab.es/~iblanes/beamer_gallery/index_by_theme.html
% You can uncomment the themes below if you would like to use a different
% one:
%\usetheme{AnnArbor}
%\usetheme{Antibes}
%\usetheme{Bergen}
%\usetheme{Berkeley}
%\usetheme{Berlin}
%\usetheme{Boadilla} %Not bad
%\usetheme{boxes}
%\usetheme{CambridgeUS}
%\usetheme{Copenhagen}
%\usetheme{Darmstadt}
%\usetheme{default}
%\usetheme{Frankfurt}
%\usetheme{Goettingen}
%\usetheme{Hannover}
%\usetheme{Ilmenau}
%\usetheme{JuanLesPins}
%\usetheme{Luebeck}
\usetheme{Madrid}
%\usetheme{Malmoe}
%\usetheme{Marburg}
%\usetheme{Montpellier}
%\usetheme{PaloAlto}
%\usetheme{Pittsburgh}
%\usetheme{Rochester}
%\usetheme{Singapore}
%\usetheme{Szeged}
%\usetheme{Warsaw}
\DeclareMathOperator*{\argmin}{arg\,min}

\title{Numerical Solutions to Mean Field Games}

% A subtitle is optional and this may be deleted
\subtitle{Models of Consensus}

%\author{Christy Graves}
\author[Christy Graves]{Christy Graves\\{\small Advisor: Ren\'{e} Carmona}}
% - Give the names in the same order as the appear in the paper.
% - Use the \inst{?} command only if the authors have different
%   affiliation.

\institute[Princeton University] % (optional, but mostly needed)
{
	Program in Applied and Computational Mathematics\\
	Princeton University
}
% - Use the \inst command only if there are several affiliations.
% - Keep it simple, no one is interested in your street address.

\date{Generals, May 4 2017}
% - Either use conference name or its abbreviation.
% - Not really informative to the audience, more for people (including
%   yourself) who are reading the slides online

%\subject{Theoretical Computer Science}
% This is only inserted into the PDF information catalog. Can be left
% out. 

% If you have a file called "university-logo-filename.xxx", where xxx
% is a graphic format that can be processed by latex or pdflatex,
% resp., then you can add a logo as follows:

% \pgfdeclareimage[height=0.5cm]{university-logo}{university-logo-filename}
% \logo{\pgfuseimage{university-logo}}

% Delete this, if you do not want the table of contents to pop up at
% the beginning of each subsection:
\AtBeginSection[]
{
  \begin{frame}<beamer>{Outline}
    \tableofcontents[currentsection,currentsubsection]
  \end{frame}
}

% Let's get started
\begin{document}

\begin{frame}
  \titlepage
\end{frame}

\begin{frame}{Outline}
  \tableofcontents
  % You might wish to add the option [pausesections]
\end{frame}

% Section and subsections will appear in the presentation overview
% and table of contents.
\section{Introduction}
%\frame{\tableofcontents[currentsection]}
\begin{frame}{Brief Intro to MFG}{N player stochastic game}
		$N$ players each make a decision (or control) $\alpha_t^i \in \mathbb{A}$ at time $t$ in an attempt to minimize their cost functions, $J^i$. For example:
		\begin{equation}
		\begin{split}
		J^i(\alpha^i,X^{1,\alpha},...X^{N,\alpha})=&\mathbb{E}\Bigg[\int_{0}^{T}f^i(t,\alpha_t^i,X_t^{1,\alpha},...X_t^{N,\alpha})dt \\
		&+g^i(X_T^{1,\alpha},...X_T^{N,\alpha}) \Bigg]
		\end{split}
		\end{equation}
		where $X_t^{i,\alpha}$ is the state of player $i$ at time $t$, given by:
		\begin{equation}
		dX_t^i=b(t,\alpha_t^i,X_t^{i,\alpha},\bar{\mu}_t)dt+\sigma(t,\alpha_t^i,X_t^{i,\alpha},\bar{\mu}_t)dW_t
		\end{equation}
		and $f^i$ and $g^i$ are the running costs and terminal costs, respectively.
\end{frame}

	\begin{frame}{Brief Intro to MFG}{Moving to Mean Field}
		\begin{itemize}
			\item {
				Assume players are symmetric.
			}
			\item {
				Consider the limit $N \rightarrow \infty$ and consider the mean field game.
		    }
			\item {
				Generic player has control $\alpha_t \in \mathbb{A}$ and state $X_t^{\alpha}$.
			}
			\item {
				Distribution of players' states over time: $\mu=(\mu_t)_{0 \leq t \leq T}$.
			}
			\item {
				Cost for generic player:
				\begin{equation}
				J(\alpha,X^{\alpha},\mu)=\mathbb{E}\Bigg[\int_{0}^{T}f(t,\alpha_t,X_t^{\alpha},\mu_t)dt+g(X_T^{\alpha},\mu_T) \Bigg]
				\end{equation}
			}
			\pause
			\item {
				Goal: find analog to Nash equilibrium
			}
		\end{itemize}
	\end{frame}



\begin{frame}{Two Consensus Problems}
	\begin{itemize}
		\item {
			Flocking: large number of birds reaching consensus on flocking velocity
		}
		\item {
			Circadian Rhythm: large number of neuronal oscillators synchronizing with each other and external stimuli (e.g. the sun)
		}
	\end{itemize}
\end{frame}

\section{State of the Art}

\subsection{Mean Field Games}

\begin{frame}{State of the Art: Mean Field Games}{Analytic Approach}
	Strategy:
	\begin{itemize}
				\item {
					Fix flow of measures: $\mu=(\mu_t)_{0 \leq t \leq T}$
				}
				\item {
					Solve stochastic control problem: find $\alpha^{\mu} \in \argmin J(\alpha,X^{\alpha},\mu)$
				}
				\item {
					Find fixed point: $\mu$ s.t. $\mathcal{L}(X_t^{\alpha^{\mu}})=\mu_t$
				}
	\end{itemize}
\end{frame}
		
\begin{frame}{State of the Art: Mean Field Games}{Analytic Approach: Stochastic Control Problem}
	\begin{itemize}
				\item {
					Define the value function:
					\begin{equation}
					\begin{split}
					V(t,x)=\inf_{\alpha \in \mathbb{A}}\mathbb{E}&\Bigg[\int_{t}^{T}f(s,\alpha_s,X_s^{\alpha},\mu_s)ds \\
					&+g(X_T^{\alpha},\mu_T) \mid X_t = x \Bigg]
					\end{split}
					\end{equation}
				}
			\end{itemize}
\end{frame}
		
\begin{frame}{State of the Art: Mean Field Games}{Analytic Approach: Stochastic Control Problem: HJB Equation}
	\begin{itemize}
				\item {
					The value function satisfies the HJB equation:
					\begin{block}{Hamilton Jacobi Bellman Equation}
						\begin{equation}
						\begin{split}
						\partial_tV(t,x)+ H(t,x, \mu_t,\partial_x V,\hat{\alpha}(t,x,\mu_t, \partial_x V))&=0 \\
						V(T,x)&=g(x,\mu_T)
						\end{split}
						\end{equation}
					\end{block}
					where  $H$ is the Hamiltonian, and $\hat{\alpha}$ is chosen to minimize the Hamiltonian.
				}
				\item {
					Derivation from the dynamic programming principle and It\^{o}'s formula
				}
			\end{itemize}
\end{frame}
		
\begin{frame}{State of the Art: Mean Field Games}{Analytic Approach: Kolmogorov (or Fokker-Planck) Equation}
	\begin{itemize}
				\item {
					Solving HJB gives us an optimal control: $\alpha(t,x)$.
				}
				\item {
					If all players use the optimal control, what is $\mathcal{L}(X_t^{\alpha})=:\mu_t$?
				}
				\item {
					This is given by the Kolmogorov (or Fokker-Planck) equation:
					\begin{block}{Kolmogorov Equation}
						\begin{equation}
						\begin{split}
						\partial_t \mu_t-\frac{1}{2} trace(\sigma \sigma^T \partial_{xx}^2 \mu_t)& \\
						+ div(b(t,x, \mu_t,\hat{\alpha}(t,x,\mu_t, \partial_x V))\cdot \mu_t)&=0 \\
						\mu_0&= \mu^0   
						\end{split}
						\end{equation}
					\end{block}
				}
				\item {
					Derivation from It\^{o}'s formula and integration by parts
				}
			\end{itemize}
\end{frame}
		
\begin{frame}{State of the Art: Mean Field Games}{Analytic Approach: Strategy Summary}
	\begin{itemize}
				\item {
					Goal: find $V, \mu$ solving HJB and Kolmogorov.
				}
				\item {
					Step 0: start with some guess for $\mu=(\mu_t)_{0 \leq t \leq T}$
				}
				\item {
					Step 1: Given $\mu$ and $V(T,x)$, solve HJB for $V$, which gives us optimal $\alpha$.
				}
				\item {
					Step 2: Given $\alpha$ and $\mu^0$, solve Kolmogorov for $\mu '$.
				}
				\item {
					Repeat Steps 1 and 2 until $\mu = \mu '$.
				}
				\pause
				\item {
					Note: existence, uniqueness, or convergence are not guaranteed. In general, need convexity of $\mathbb{A}$, $f$, and $g$.
				}
				\pause
				\item {
					Another note: this system of PDEs is highly coupled, nonlinear, with one PDE in the forward direction and the other in the backward direction.
				}
	\end{itemize}
\end{frame}

\begin{frame}{State of the Art: Mean Field Games}{Existence \& Uniqueness}
	\begin{itemize}
		\item {
			Cauchy-Lipschitz theory only holds for small T.
		}
		\item {
			Existence
			\begin{itemize}
				\item In general, need Lipschitz coefficients and cost functions, non-degenerate diffusion coefficient.
				\item Relies on Schauder's Fixed Point Theorem.
			\end{itemize}
		}
		\item {
			Uniqueness
			\begin{itemize}
				\item Assumming no common noise, in general, need monotone cost functions.
				\begin{itemize}
				\item Lasry Lions Monotonicity
				\item $L$-Monotonicity
				\end{itemize}
			\end{itemize}
		}
	\end{itemize}
\end{frame}

\begin{frame}{State of the Art: Mean Field Games}{Some Results}
	\begin{itemize}
		\item {
			Linear-Quadratic Games.
		}
		\begin{itemize}
			\item Stochastic maximum approach leads to matrix Ricatti equation.
		\end{itemize}
		\item Some results when cost is local.
		\begin{itemize}
			\item Cost is local and linear: Swiecicki et. al. \textit{An [imaginary time] Schr{\"o}dinger approach to mean field games.}
		\end{itemize}
		\item With common noise: Cardaliaguet et. al. \textit{The master equation and the convergence problem in mean field games.}
		\item Nonlocal cost of particular form: Graber \& Bensoussan \textit{Existence and Uniqueness of Solutions for Betrand and Cournot mean field games.}
	\end{itemize}
\end{frame}

\begin{frame}{State of the Art: Mean Field Games}{Open Problems}
	\begin{itemize}
		\item {
			More general existence and uniqueness results.
		}
		\item {
			Problems with non-convex cost functions.
		}
		\item {
			Problems with non-local cost functions.
		}
	\end{itemize}
\end{frame}

\subsection{Flocking: Previous Work}

\begin{frame}{State of the Art: Flocking}{Previous Work}
	Cucker-Smale (CS)
	\begin{equation}
				\begin{split}
				x'_i(t)&=v_i(t)\\
				v'_i(t)&=\frac{1}{N}\sum_{j=1}^N\frac{v_j(t)-v_i(t)}{(1+\Vert x_j(t)-x_i(t) \Vert^2)^\beta}
				\end{split}
	\end{equation}
	\begin{itemize}
		\item {
			Unconditional flocking when $\beta < 1/2$.
		}
		\item {
			When $\beta \geq 1/2$, flocking occurs for sufficiently small values of $\sum_{i \neq j}\Vert x_i(0)-x_j(0)\Vert^2$ and $\sum_{i \neq j}\Vert v_i(0)-v_j(0)\Vert^2$.
		}
	\end{itemize}
\end{frame}

\begin{frame}{State of the Art: Flocking}{Aside on $\beta$ Parameter}
	\begin{itemize}
		\item {
			My first point.
		}
		\item {
			My second point.
		}
	\end{itemize}
\end{frame}

\begin{frame}{State of the Art: Flocking}{Previous Work}
	Kinetic CS (Ha and Tadmor)
	\begin{itemize}
		\item {
			Mean field limit of CS.
		}
		\item {
			PDE of McKean-Vlasov type:
		}
		\begin{equation}
		\partial t \mu_t+v \partial_x \mu_t+\partial_v \left[\xi (x,v,\mu_t) \mu_t \right]=0
		\end{equation}
		where
		\begin{equation}
		\xi (x,v,\mu_t)=\int_{\mathbb{R}^{2d}} \frac{v'-v}{(1+\Vert x'-x \Vert^2)^\beta}\mu_t(x',v')dx'dv'
		\end{equation}
	\end{itemize}
	Results
	\begin{itemize}
		\item For compactly supported $\mu_0(x,v)$, time-asymptotic flocking when $\beta \leq 1/4$ (later extended to $\beta \leq 1/2$ by Ha and Liu).
	\end{itemize}
\end{frame}

\begin{frame}{State of the Art: Flocking}{Previous Work}
	Normalized Weights Kinetic CS (?)
	\begin{itemize}
		\item {
			Normalize weights so that the control is a true weighted average of $v'-v$.
		}
		\item {
			PDE of McKean-Vlasov type:
		}
		\begin{equation}
		\partial t \mu_t+v \partial_x \mu_t-v \partial_v \mu_t-\mu_t+\phi (x,\mu_t)\partial_v \mu_t=0
		\end{equation}
		where
		\begin{equation}
		\phi (x,\mu_t)=\frac{\int_{\mathbb{R}^{2d}} \frac{v'}{(1+\Vert x'-x \Vert^2)^\beta}\mu_t(x',v')dx'dv'}{\int_{\mathbb{R}^{2d}} \frac{1}{(1+\Vert x'-x \Vert^2)^\beta}\mu_t(x',v')dx'dv}
		\end{equation}
	\end{itemize}
	Results???
\end{frame}



\begin{frame}{State of the Art: Flocking}{Previous Work}
	Mean Field Game Analog of CS (Nourian, Caines, and Malhalm\'{e})
				\begin{equation}
				\begin{split}
				dx(t)&=v(t)dt\\
				dv(t)&=\alpha_tdt+\sigma dW
				\end{split}
				\end{equation}
				Where $\alpha$ is chosen to minimize:
				\begin{equation}
				\begin{split}
				J(\alpha)&=\limsup_{T \rightarrow \infty} \frac{1}{T} \int_0^T \Psi(x,v,\mu_t)+\alpha_t^2 dt \\
				\Psi(x,v,\mu_t)&=\Vert\int_{\mathbb{R}^{2d}}\frac{v'-v}{(1+\Vert x'-x \Vert^2)^\beta}\mu_t(x',v')dx'dv' \Vert^2
				\end{split}
				\end{equation}
	Results:
	\begin{itemize}
		\item {
			Exact solution when $\beta=0$. (Linear Quadratic Game.)
		}
		\item {
			As $t\rightarrow \infty$, $v \sim N(\bar{v}_0,\sigma^2/2)$
		}
	\end{itemize}
\end{frame}

\begin{frame}{State of the Art: Flocking}{Previous Work Summary}
	Finitely many birds, no optimization:
	\begin{center}
	\begin{tabular}{|c|c|c|}
	 \hline
	 Authors & Noise & Interaction Type  \\ \hline
	 Cucker, Smale 2007& no & spatial weights \\ \hline
	 Ha, Liu 2009& no&spatial weights \\ \hline
	 Martin, Girard 2010 & no&unweighted $(\Vert x'-x\Vert^2 <R)$ \\ \hline
	 Cucker, Dong 2010 & no&spatial weights \& repulsion \\ \hline
	 Cucker, Mordecki 2008 & yes&spatial weights \\ \hline
	 Ha, Lee, Levy 2009 & yes&spatial weights \\ \hline
	 Ton, Linh, Yagi 2015 & yes& noisy spatial weights \\
	 \hline
	\end{tabular}
	\end{center}
\end{frame}

\begin{frame}{State of the Art: Flocking}{Previous Work Summary}
	Justification of Mean Field Limit (still no optimization):
	\begin{center}
		\begin{tabular}{|c|c|c|}
			\hline
			Authors & Noise & Result  \\ \hline
			Ha, Tadmor 2008& no & flocking for initially compact data \\ \hline
			Carrillo, et. al. 2010& no & exponentially fast flocking \\ \hline
			Bolley, Ca\~nizo, Carrillo 2010 & yes& existence and uniqueness \\
			\hline
		\end{tabular}
	\end{center}
	All of these results are for the spatial weights.
\end{frame}

\begin{frame}{State of the Art: Flocking}{Previous Work Summary}
	Mean field:
	\begin{center}
		\begin{tabular}{|c|c|c|}
			\hline
			Authors & Noise & Interaction Type  \\ \hline
			Ca\~nizo, et. al. 2010& no & spatial weights \& A/R \\ \hline
			Karper, et. al. 2013& yes & spatial weights \\
			& & existence of weak solutions \\ \hline
			Piccoli, et. al. 2015& no & spatial weights \& sparse external force \\ \hline
			Barbaro, et. al. 2016& yes & normalized spatial weights \& A/R \\
			& & results only for $\beta=0$ \\ \hline
			Nourian, et. al. 2010 & yes& optimization (MFG), spatial weights \\
			& & results only for $\beta=0$ \\
			\hline
		\end{tabular}
	\end{center}
\end{frame}

\begin{frame}{State of the Art: Flocking}{Open Problems}
	\begin{itemize}
		\item {
			Noisy Kinetic or MFG analog of CS Flocking when $\beta > 0$, with and without normalized spatial weights.
		}
	\end{itemize}
\end{frame}

\section{My Progress}

\subsection{Flocking Problem Formulation}

\begin{frame}{Flocking Problem Formulation}
	\begin{itemize}
		\item {
			For now, consider the MFG formulation of Nourian et. al. but with finite horizon T.
		}
						\begin{equation}
						\begin{split}
						dx(t)&=v(t)dt\\
						dv(t)&=\alpha_tdt+\sigma dW
						\end{split}
						\end{equation}
						Where $\alpha$ is chosen to minimize:
						\begin{equation}
						\begin{split}
						J(\alpha)&=\mathbb{E} \int_0^T \Psi(x,v,\mu_t)+\alpha_t^2 dt \\
						\Psi(x,v,\mu_t)&=\Vert\int_{\mathbb{R}^{2d}}\frac{v'-v}{(1+\Vert x'-x \Vert^2)^\beta}\mu_t(x',v')dx'dv' \Vert^2
						\end{split}
						\end{equation}
	\end{itemize}
\end{frame}

\begin{frame}{Flocking Problem Formulation}
	\begin{itemize}
		\item {
			For now, consider the MFG formulation of Nourian et. al. but with finite horizon T.
		}
		\item {
			The solution is given by the HJB and Kolmogorov equations:
			\begin{equation}
			\begin{split}
			\partial_t V+\frac{\sigma^2}{2}\partial^2_{vv} V+ v \partial_x V+ \alpha \partial_v V+\alpha^2+\Psi(x,v,\mu)&=0 \\
			V(T,x,v)&=0 \\
			\partial_t \mu_t-\frac{\sigma^2}{2}\partial^2_{vv} \mu_t + v \partial_x \mu_t +div_v(\alpha_t \mu_t) &=0 \\
			\mu(0,x,v)=\mu^0(x,v)& \\
			\alpha=-\frac{1}{2}\partial_v V&
			\end{split}
			\end{equation}
		}
	\end{itemize}
\end{frame}

\subsection{Deriving Stable Scheme \& Boundary Conditions}

\begin{frame}{Numerical Approach}
		\begin{itemize}
			\item {
				To solve numerically, one common method is finite differences.
			}
			\item {
				The unknown function, $f$, will be approximated on a lattice of points:
				\begin{equation}
				f(t_i,x_j) \approx f(i \cdot \Delta t,j \cdot \Delta x)
				\end{equation}
			}
			\item {
				All derivatives are replaced by finite differences.
			}
			\item {
				For example, the heat equation:
				\begin{equation}
				\partial_t f= \partial_{xx}^2 f
				\end{equation}
			}
			\item {
				becomes:
				\begin{equation}
				\frac{f(t_{i+1},x)-f(t_i,x)}{\Delta t}=\frac{f(t_i,x_{j+1})-2f(t_i,x_j)+f(t_i,x_{j-1})}{\Delta x^2}
				\end{equation}
			}
		\end{itemize}
\end{frame}

\begin{frame}{Numerical Approach, continued}
	Following the analytic approach, our numerical approach is the following:
	\begin{itemize}
			\item {
				Goal: find grid functions $V(t_k,x_i,v_j)$ and $\mu_{t_k}(x_i,v_j)$ solving finite difference equations for HJB and Kolmogorov.
			}
			\item {
				Step 0: start with some guess for $\mu=(\mu_{t_k}(x_i,v_j))_{0 \leq t_k \leq T}$
			}
			\item {
				Step 1: Given $\mu$ and $V(T,x)$, solve finite difference HJB explicitly backwards in time for $V$, which gives us optimal $\alpha$.
			}
			\item {
				Step 2: Given $\alpha$ and $\mu^0$, solve finite difference Kolmogorov explicitly forwards in time for $\mu '$.
			}
			\item {
				Repeat Steps 1 and 2 until $\mu \approx \mu '$.
			}
	\end{itemize}
\end{frame}

\begin{frame}{Numerical Approach: Complications}
		\begin{itemize}
			\item {
				Curse of dimensionality: number of lattice points grows exponentially with the dimension of the state space.
				
			}
			\item {
				Stability: $\Delta t$ can't be too big (CFL condition).
				\begin{center}
					$\Delta x$ smaller and b (drift) larger $\implies$ need smaller $\Delta t$.
				\end{center}
			}
			\item {
				Accuracy: use upwind scheme for 1st order spatial derivatives
			}
			\item {
				What boundary conditions should be used?
			}
		\end{itemize}
\end{frame}

	\begin{frame}{Numerical Approach: Stability and Accuracy}
		\begin{itemize}
			\item CFL condition:
			\begin{equation} 
			\Delta t \leq \frac{1}{2(\frac{\sigma^2}{\Delta x^2}+\frac{|b|}{\Delta x})}
			\end{equation}
			\item Upwind scheme:
			\begin{equation}
			[Lf]=b \cdot \frac{\partial f}{\partial x} + \frac{\sigma^2}{2}\frac{\partial^2 f}{\partial x^2}
			\end{equation}
			\begin{equation}
			[Lf](i)=\sum_{j} L(i,j) f(x_j)
			\end{equation}
			Choose forward or backward differences such that:
			\begin{enumerate}
				\item $L(i,j)\geq 0$ when $i \neq j$.
				\item $L(i,i)=-\sum_{j\neq i} L(i,j)$
			\end{enumerate}
			
		\end{itemize}
	\end{frame}
	
	\begin{frame}{Numerical Approach: Upwind Scheme}
		For example:
		\begin{equation}
		\left[b \cdot \frac{\partial f}{\partial x} \right](x_i)=b^{+} \cdot \frac{f(x_{i+1})-f(x_i)}{\Delta x}-b^{-} \cdot \frac{f(x_{i})-f(x_{i-1})}{\Delta x}
		\end{equation}
		\begin{equation}
		L(i,i+1)=b^{+} \cdot \frac{1}{\Delta x} \geq 0
		\end{equation}
		\begin{equation}
		L(i,i-1)=-b^{-} \cdot \frac{-1}{\Delta x} \geq 0
		\end{equation}
		\begin{equation}
		L(i,i)=b^{+} \cdot \frac{-1}{\Delta x}-b^{-} \cdot \frac{1}{\Delta x} = -[L(i,i+1)+L(i,i-1)]
		\end{equation}
	\end{frame}
	
	\begin{frame}{Numerical Approach: Stability of $\mu$}
		\begin{itemize}
			\item CFL + Upwind $\rightarrow$ Solution to Kolmogorov is a prob. measure, i.e.
			\begin{equation}
			\begin{split}
			\mu(t_n,x_i) & \geq 0 \\
			\sum_i \mu(t_n,x_i) &= 1
			\end{split}
			\end{equation}
			\item Proof:
			\begin{equation}
			\frac{\mu(t_{n+1},x_i)-\mu(t_{n},x_i)}{\Delta t}=\sum_{j} L(j,i) \mu(t_n,x_j)
			\end{equation}
			\begin{equation}
			\begin{split}
			\mu(t_{n+1},x_i)&=\mu(t_{n},x_i)+ \Delta t \cdot L(i,i)\mu(t_n,x_i) + \Delta t \cdot \sum_{j\neq i} L(j,i) \mu(t_n,x_j)  \\
			& \geq \mu(t_{n},x_i) (1+ \Delta t \cdot L(i,i))
			\end{split}
			\end{equation}
			$\mu(t_{n+1},x_i) \geq 0$ if $\Delta t \cdot| L(i,i)| \leq 1$ (CFL Condition)
		\end{itemize}
	\end{frame}
	
	\begin{frame}{Numerical Approach: Stability of $\mu$ (continued)}
		\begin{equation}
		\begin{split}
		\sum_i \mu(t_{n+1},x_i)=& \sum_i \mu(t_{n},x_i)+\Delta t \cdot \sum_i \sum_{j} L(j,i) \mu(t_n,x_j) \\
		& = 1 + \Delta t \cdot \sum_j \Big(\sum_i L(j,i)\Big) \mu(t_n,x_j) \\
		& = 1
		\end{split}
		\end{equation}
	\end{frame}

\begin{frame}{Numerical Approach: Boundary Behavior}
	\begin{itemize}
		\item {
			The domain of the original problem (in 1 dimension) is $\mathbb{R}^{2d}$. For numerics, we are restricted to a finite domain.
		}
		\item {
			Let the domain of $v$ be $D_v =[-v_{max},v_{max}]$.
			ToDo: less text
			\begin{itemize}
				\item If birds were 'killed' at the boundary, the total mass of the system would decrease in time, which is undesirable.
				\item If the boundary was periodic, then birds that have a large positive velocity would be replaced by birds with a large negative velocity, which is unrealistic.
				\item Intuitively, if a bird reaches a physical upper limit on it's velocity, it would have to decrease it's velocity. Thus, we choose reflecting boundary conditions at the boundary $\partial D_v$.
				\item Nonetheless, $v_{max}$ will be chosen large enough such that only a small amount of the total mass can reach the boundary $\partial D_v$, limiting the effect of our choice of boundary condition for $\partial D_v$.
			\end{itemize}
		}
	\end{itemize}
\end{frame}

\begin{frame}{Numerical Approach: Boundary Behavior, continued}
	\begin{itemize}
		\item {
			Let $R \in \mathbb{R}$ such that $x_0 \in [-R,R]$ for all birds. Then $x_t \in [-R-Ty_{ax},R+T y_{max}]$ $\forall t$.
		}
		\item {
			Let the domain of $x$ be $D_x =[-R-Ty_{max}-\epsilon,R+T y_{max}+\epsilon]$. Then no birds will ever reach the boundary $\partial D_x$. Thus, the choice of boundary condition for $\partial D_x$ should not affect these results, and we can arbitrarily choose reflecting boundary conditions for $\partial D_x$.
		}
	\end{itemize}
\end{frame}

\begin{frame}{Numerical Approach: Boundary Conditions}
	\begin{itemize}
		\item {
			Now that we have established boundary behavior, we need to establish boundary conditions for the HJB and Kolmogorov equations.
		}
		\item {
			Reflecting on the boundaries corresponds to a von Neumann boundary condition for the HJB:
			\begin{equation}
			\begin{split}
			\partial_v V(t,x,y_{min})&=0 \\
			\partial_v V(t,x,y_{max})&=0 \\
			\partial_x V(t,x_{min},y)&=0 \\
			\partial_x V(t,x_{max},y)&=0
			\end{split}
			\end{equation}
		}
		\item {
			We also need an approximation for $\partial^2_{vv} V(t,x,y)$ for $y=y_{min},y_{max}$.
		}
	\end{itemize}
\end{frame}

\begin{frame}{Numerical Approach: Boundary Conditions, continued}
	\begin{itemize}
		\item {
			By making the approximation $V(t,x,y_{min}-\Delta v) \approx V(t,x,y_{min}+\Delta v)$ (based on the von Neumann boundary), we can approximate:
			\begin{equation}
				\partial^2_{vv} V(t,x,y_{min})\approx \frac{2V(t,x,y_{min}+\Delta v)-2V(t,x,y_{min})}{\Delta v^2}
			\end{equation}
			and similarly for $y_{max}$.
		}
		\item {
			Now, we have fully specified the discrete operator $L$ such that the finite difference HJB equation is given by:
			\begin{equation}
			\partial_t V+L[V]=0
			\end{equation}
		}
		\item {
			The boundary conditions for the finite difference Kolmogorov equation are already determined by our choice for $L$:
			\begin{equation}
			\partial_t \mu_t=L^*[\mu_t]
			\end{equation}
		}
	\end{itemize}
\end{frame}

\subsection{Numerical Results}

\begin{frame}{Numerical Results}{MFG of flocking, unnormalized weights}
	\begin{itemize}
		\item {
			ToDo: show variance increases with t for $\beta >0$.
		}
		\item {
			Lead into motivation for using normalized weights
		}
	\end{itemize}
\end{frame}

\begin{frame}{Numerical Results}{MFG of flocking, normalized weights}
	\begin{itemize}
		\item {
			ToDo: show variance decreases/stabilizes
		}
		\item {
			Explain how the relationship of variance with $\beta$, $\lambda$, and $\sigma$ is not clear. To simplify the problem, let's return to the Kinetic-CS model. (Question: does variance increase for unnormalized Kinetic-CS? Not sure if I ever ran this.)
		}
	\end{itemize}
\end{frame}

\begin{frame}{Numerical Results}{Kinetic-CS flocking, normalized weights}
	\begin{itemize}
		\item {
			ToDo: show variance goes to $\sigma^2/2$.
		}
	\end{itemize}
\end{frame}

\begin{frame}{Numerical Results}{Verifying Code}
	\begin{itemize}
		\item {
			ToDo: search 'verify' in Notes
		}
		\item {
			Mass is conserved. Mass is nonnegative. (Put in better wording and using equations)
		}
	\end{itemize}
\end{frame}

\begin{frame}{Conjecture}
	\begin{block}{Conjecture}
		For $\mu_0(x,v)$ with compact support, the solution to noisy Kinetic Cucker-Smale flocking with normalized weights has the following time-asymptotic behavior:
			\begin{equation}
			\lim_{t \rightarrow \infty} Var(v_t)=\frac{\sigma^2}{2} 
			\end{equation}
			Or equivalently,
			\begin{equation}
			\lim_{t \rightarrow \infty} \mathbb{E}(v_t-\mathbb{E}v_t)^2=\frac{\sigma^2}{2}
			\end{equation}
	\end{block}
\end{frame}

\subsection{Analytical Progress}

\begin{frame}{Analytical Progress}{Direct Calculation}
	\begin{itemize}
		\item {
			ToDo: g(t) going to 0, lose paper in blue book.
		}
		\item {
			My second point.
		}
	\end{itemize}
\end{frame}

\begin{frame}{Analytical Progress}{Expansion}
	\begin{itemize}
		\item {
			ToDo: Motivate small $\beta$
		}
		\item {
			Assume $\mu_0(x,v)$ is symmetric about some $(\bar x_0,\bar v_0)$
		}
		\item {
			Show that without loss of generality, can assume $(\bar x_0,\bar v_0)=(0,0)$ (because can translate solution by $(t\bar v_0,\bar v_0))$
		}
	\end{itemize}
\end{frame}

\begin{frame}{Analytical Progress}{Expansion, continued}
	\begin{itemize}
		\item {
			ToDo: Show expansion and 0th and 1st order systems.
		}
	\end{itemize}
\end{frame}

\begin{frame}{Analytical Progress}{Expansion Summary}
	\begin{itemize}
		\item {
			ToDo: Show general strategy for building solution
		}
	\end{itemize}
\end{frame}

\begin{frame}{Analytical Progress}{Expansion: General Initial Condition}
	\begin{itemize}
		\item {
			ToDo: Show problem term that keeps us from using the approach from before
		}
	\end{itemize}
\end{frame}


\section{Future Directions}

\subsection{Numerics}

\subsection{Ideas for Proving Conjecture}

\begin{frame}{Future Directions}{Ideas for Proving Conjecture}
	\begin{itemize}
		\item {
			My first point.
		}
		\item {
			My second point.
		}
	\end{itemize}
\end{frame}

\begin{frame}{Future Directions}{Numerics}
	\begin{itemize}
		\item {
			My first point.
		}
		\item {
			My second point.
		}
	\end{itemize}
\end{frame}

\subsection{New Problem: Circadian Oscillators}

\begin{frame}{Future Directions}{New Problem: Circadian Oscillators}
	\begin{itemize}
		\item {
			My first point.
		}
		\item {
			My second point.
		}
	\end{itemize}
\end{frame}

% Placing a * after \section means it will not show in the
% outline or table of contents.
\section*{Summary}
\begin{frame}{Summary}
	\begin{itemize}
		\item {
			My first point.
		}
		\item {
			My second point.
		}
	\end{itemize}
\end{frame}

\end{document}


